User sessions have been used as behavioral measures of human-computer interaction for almost two decades, and for this reason, strategies for session identification of log data has been the extensively studied\cite{eickhoff2014lessons}.

Cooley et al.\cite{cooley1999data} and Spiliopoulou et al.\cite{spiliopoulou2003framework} contast two primary strategies for identifying sessions from activity logs: ``navigation-oriented heuristics'' and ``time-oriented Heuristics''.  Time-oriented heuristics refers to the assignment of a inactivity threshold between logged activites to serve as a session delimiter.  The assumption here is that if there is a break between a user's actions that is sufficiently long, it's likely that the user is no longer \emph{active}, the session is assumed to have ended, and a new session is created when the next action is performed. This is the most commonly used approach to identify sessions, with 30 minutes as the most commonly used threshold\cite{spiliopoulou2003framework,eickhoff2014lessons,ortega2010differences}.  Both threshold and approach appear to originate in a 1995 paper by Catledge \& Pitkow\cite{catledge1995characterizing} that used client-side tracking to identify browsing behaviour. In their work, they reported that the mean time between user observed user events in their data was 9.3 minutes.  They choose to add 1.5 standard deviations to that mean to achieve a 25.5 minutes inactivity threshold.  Over time this proposed inactivity threshold has gradually been smoothed out to 30 minutes.  The utility of the 30-minute threshold is widely debated; Mehrzadi \& Feitelson (2012) \cite{mehrzadi2012onextracting} found that 30 minutes produced artefacts around long sessions, and could find no clear evidence of a natural session inactivity threshold\footnote{Note that this conclusion was reached using the same AOL search dataset that we analyze in this paper.}, while Jones \& Klinkner\cite{jones2008beyond} found the 25.5 minute threshold ``no better than random'' in the context of search tasks. Other thresholds have been proposed, but Montgomery and Faloutsos\cite{montgomery2001identifying} concluded that the actual threshold chosen made little difference to its accuracy.

Referer-based reconstruction involves taking the referers and URLs associated with each request by a user, and chaining them together. When a user begins navigating without a referer, they have started a session; when a trail can no longer be traced to that request based on the referers and URLs of subsequent requests, the session has ended.  This approach was pioneered by Cooley et al in 2002\cite{cooley1999data}.  While it demonstrated utility in identifying ``tasks'', and has been extended by Nadjarbashi-Noghani et al.\cite{nadjarbashi2004improving} it shows poor performance on sites with framesets due to implicit assumptions about web architecture\cite{berendt2003impact}. The sheer complexity of this strategy and it's developmental focus on \emph{task} over \emph{session} make it unsuitable as a replacement for time-oriented heuristics in practical web analytics.

In our this work, we'll challenge the conclusions of prior works' assertions (1) that no reasonable cutoff is from the empirical data and (2) that a global inactivity threshold is inappropriate as a session identification strategy.  To our knowledge, we are the first to apply a general session identification methodology to a large collection of datasets and conclude that not only are global inactivity thresholds an appropriate strategy for session identification, but also that, for most user-initiated actions, an inactivity threshold of 1 hour is appropriate.
