Our work stands in the face of previous work ...

* We propose a rule of thumb and a simple methodology.  Our analysis suggests that setting an inactivity threshold at one hour may be robust to new datasets.  However, we still advise that any new application of session identification using an inactivity threshold is preceeded by a plot of a log histograph of inter-activity times and visual inspection for a natural valley between 1 minute and 1 day. 


* On the nature of human activity, strong regularities of inter-activity time.
** Activity Theory?  Maybe these clusters represent operation, action and activity session.
* What this might imply for the design of systems.
** If human behavior is well represented by a sequence of conscious activities the regularities in time between events hold...
** System designers can take advantage of this by designing systems that afford operations, actions and activity at timescales that humans will feel to be natural.
** Indeed, we suspect that activities that force users to devate from these patterns may be frustrating or may otherwise limit their ability to function at full capacity.
