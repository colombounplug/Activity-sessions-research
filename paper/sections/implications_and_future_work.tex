In this paper, we have challenged previous literature that suggest no apparent inactivity threshold exists for identifying user session from activity logs.  From our reuslts this, we propose a simple, yet apparently robust, rule of thumb and a methodology for checking this rule in other datasets.  The rule of thumb is easy to apply; our analysis suggests that setting an inactivity threshold to demarcate the end of a session at \emph{one hour} will be appropropriate for most kind of activity log analysis.

We suspect that this strategy will be robust to new datasets since it is (1) grounded in empirical observations of a natural valley in activity times that corresponds to our intuitions about user activities and (2) holds constant across a wide range of systems and activity types. Even when our threshold detection strategy deviated from one hour, the devations were relatively small given the scale of activities, and in some cases, this deviation could be explained by limitations in the data used to fit our models.  However, we still advise that any new application of session identification using an hour as an inactivity threshold is preceeded by a plot of a histograph of log-scaled inter-activity times and visual inspection for a natural valley between 1 minute and 1 day to confirm its suitability.

These results and our recommendations stand in the face of a long and nuanced discussion of the nature of user sessions as can be extracted from logged interactions with a computer system.  We place our criticisms of previous work into two categories: (1) previous empirical work did not attempt to look for log-normally distributed patterns and therefore concluded that no obvious separation between within- and between session inter-activity times exist\cite{mehrzadi2012onextracting}\cite{catledge1995characterizing} and (2) other work exploring \emph{task driven} behavior conflates ``task'' with ``session''.  We challenge (1) on the basis of the clear trends represented in the results of this work and (2) by drawing a distinction between goal-directed tasks (\emph{action} in Activity Theory) and activity sessions which often represent a collection of heterogenious goal-directed tasks.

Further, given the strong regularities we see between between different types of human-computer interactions, our results suggest something more fundimental about human activity itself.  As discussed in section \ref{sec:human_activities}, Activity Theory(AT) conceptualizes human consciousness as a sequence of \emph{activities} which represent a heirarchical relationship with \emph{actions} and \emph{operations}.  We suspect that the fact that \emph{operations} and \emph{actions} must be performed in a sequence may suggest a natural temporal rhythm.  While it's hard to say conclusively, we suspect that the ``short\_within'' clusters we observe represent \emph{operation}-level events, the ``within'' clusters represent \emph{action}-level events, and the ``between'' clusters represent \emph{activity}-level events.

If this application of AT to the observed patterns is accurate, this could have substantial implications for the design of systems.  System designers may be able to take advantage of the regularities observed by constructing systems that afford operations, actions and activity at timescales that humans will feel to be natural.  Our analysis suggests that operations should exist at the timescale of about 5-20 seconds, actions should be completable at a timescale of 1-7 minutes and activities should be supported at daily to weekly time intervals.  We suspect that systems that do not allow users to work at these time scales may be frustrating or may otherwise limit their ability of their users to function at full capacity.

These ruminations about human behavior and its manifestation in well designed systems are only speculation at this point.  New work will need to be done to explore whether our predictions hold and whether any effect is truely of substantial effect.
